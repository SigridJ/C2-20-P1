\section{Versioner og approksimationer af TSP}
Udover at TSP ikke har nogen polynomiel algoritme, der kan løse problemet, har den generelle version af problemet heller ikke en $k$-approksimation.
For at vise dette resultat, er det nødvendigt at vise, at en sådan $k$-approksimation giver anledning til en løsning til Hamiltonian Path Problem (HPP).
HPP er et kendt problem, som blev beskrevet i Kapitel \ref{chap:paths_and_circuts}, og problemet er $NP$-fuldstændigt, jf. Sætning \ref{HPP}.

\begin{thm} \label{thm:TSP_HPP}
	Hvis der eksisterer en $k$-approksimation for TSP, følger det, at der eksisterer en polynomiel algoritme til at løse HPP. 
\end{thm}

\begin{proof}
	Det vil blive vist, at hvis der eksisterer en $k$-approksimation for TSP, vil dette give anledning til en algoritme, der løser HPP i polynomiel tid.

	Antag at der eksisterer en $k$-approksimation, der løser TSP.
	Lad $G=(V,E)$ være en simpel graf, hvori det ønskes at vide, om der eksisterer en Hamiltonkreds.
	Lad nu $G'$ være en komplet graf med præcis de samme knuder som $G$, og lad de kanter $G'$ har til fælles med $G$ have vægten $1$, mens alle de resterende får vægten $kn$, hvor $n$ er antallet af knuder.

	$k$-approksimationen køres nu med $G'$.
	Hvis der eksisterer en Hamiltonkreds i $G$, er den mindst mulige samlede vægt for en Hamiltonkreds i $G'$ lig $n$, da alle kanternes vægt i kredsen er lig $1$.
	Da må $k$-approksimationen finde en samlet vægt på mindre end eller lig $kn$.
	Hvis ikke der eksisterer en Hamiltonkreds i $G$, må den mindst mulige samlede vægt for en Hamiltonkreds i $G'$ være skarpt større end $kn$, da der må være mindst én kant i kredsen hvis vægt er lig $kn$.
	Da må også $k$-approksimationen finde en vægt, der er skarpt større end $kn$.

	Altså kan der ud fra $G$ konstrueres en ny graf, og hvis $k$-approksimationen køres med denne graf, kan værdien, den returnerer, altid bruges til at bestemme, om der eksisterer en Hamiltonkreds i $G$.
	Siden $k$-approksimationen per Definition \ref{def:apk} er af polynomiel tid, må dette også være en løsningsalgoritme til HPP af polynomiel tid.
\end{proof}

\begin{tcolorbox}
	\begin{cor} \label{thm:korollar}
		Der eksisterer ikke en $k$-approksimationsalgoritme for TSP når $k \geq 1$, medmindre $P = NP$.
	\end{cor}
\end{tcolorbox}
\begin{proof}
	Antag for modstrid at der eksisterer en $k$-approksimation for TSP.
	Jf. Sætning \ref{thm:TSP_HPP} må der da eksistere en polynomiel algoritme, der løser HPP. 
	Fra Sætning \ref{HPP} fås det, at HPP er $NP$-fuldstændigt, hvilket vil være en modstrid, medmindre $P=NP$.
\end{proof}

Selvom der ikke eksisterer en algoritme til at finde den optimale løsning på TSP i polynomiel tid, er det muligt med algoritmer at approksimere en optimal løsning i polynomiel tid. Approksimationer kan blive mere præcise ved at restringere problemet.
Det er typisk en af de følgende tre restriktioner, der vil blive brugt. \citep{complex} 
\begin{itemize}[noitemsep]
	\item Symmetrisk TSP, hvor alle vægte er symmetriske, altså at $d(v_i, v_j) = d(v_j, v_i)$, for to knuder $v_i$ og $v_j$ i grafen. Kun en ikke-simpel graf giver anledning til denne version af TSP.
	\item Metrisk TSP, hvor alle tripler af kanters vægt i grafen opfylder trekantsuligheden.
	\item Euklidisk TSP, hvor knuderne er punkter i et euklidisk rum, ${\mathbb{R}}^d$, og kanternes vægt er afstanden mellem knuderne, de forbinder.
\end{itemize}

Den metriske version af problemet er en rimelig restriktion af problemet.
Typisk vil eksempelvis et vejnet opfylde trekantsuligheden for alle tripler af knuder, da den mest direkte vej mellem to byer, der er forbundet, som regel aldrig er længere end hvis der tages en rute gennem en tredje by.
Dog er forskellen på metrisk TSP og euklidisk TSP, at den euklidiske version ikke vil kunne passe på et vejnet, hvor vejene svinger, og der bliver da længere end den direkte afstand mellem to byer.

En af grundene til at restringere det gennerelle TSP er, at den metriske version har en $k$-approksimationsalgoritme, til trods for at TSP ikke har.

\begin{tcolorbox}
	\textbf{Metrisk TSP} \quad Lad $G=(V,E)$ være en simpel, komplet, vægtet graf. Der skal gælde trekantsuligheden
	\begin{align*}
		d(u,v) \leq d(v,w) + d(w,u)
	\end{align*}
	for alle $u,v,w \in V$.

	Find en Hamiltonkreds i denne graf med den mindste samlede vægt.
\end{tcolorbox}

Til denne version af TSP vil der blive undersøgt en mulig approksimeringsalgoritme.
