\chapter{Algoritmer}
Mange matematiske problemer kan ikke løses ved hjælp af en simpel beregning, men kræver i stedet en række af skridt der endeligt giver en løsning til problemet. 
En beskrivelse af den nødvendige sekvens af skridt, kaldes en algoritme. 


\begin{defn}
En algoritme er et endeligt sæt af præcise instruktioner, som beskriver udførelsen af en beregning eller løsning af et problem.
\end{defn}

I denne rapport vil vi bruge pseudokode til at beskrive hvordan man kan opstille en algoritme i computersprog. 
En pseudokode kan ikke skrives direkte ind i et programmeringsprogram, men kan let oversættes til et ønsket programmeringssprog og kan let læses. 

Skal man finde den største værdi i en liste, kan dette gøres nemt ved hjælp af en algoritme. Selvom det kan virke meget simpelt at læse en liste igennem og finde det største tal, kan det godt blive problematisk og tidskrævende hvis der er tale om en længere liste, derfor er en algoritme smart da den hurtigt kan gennemgå lange lister.
En sådan algoritme kan udføres, ved et sætte det første element lig med maks, dernæst sammenlignes denne værdi med det næste element, som overtager maks hvis det er større end det daværende maks. 
Er det næste element mindre end det nuværende maks sker der ikke noget og man går videre til det næste element på listen. 
Dette gentages indtil man har gennemgået hele listen, hvorefter maks returneres.
Herunder ses et eksempel på hvordan en søgealgortime kan skrives op med pseudokode:


\begin{algorithm}
\caption{Find maksimalt element i en liste}
\label{find_maks}
\textbf{procedure} $ maks(a_1, a_2, ... a_n) $

$ maks:=a_1 $ \\
\textbf{for} $i :=2$ \textbf{til} $n$ \\
$\-$ $\-$ $\-$ $\-$ $\-$ $\-$
\textbf{hvis} $maks<a_i$ \textbf{så}
$maks:=a_i$ \\
\textbf{returner} $maks \lbrace maks$ er det største element $\rbrace$
\end{algorithm}

Et andet problem der kan løses ved hjælp af algoritmer, er sorteringen af elementerne i en liste. 
Til dette findes der flere forskellige algoritmer, men her vil kun "bubblesort$"$ blive beskrevet. 
I bubblesort algoritmen undersøger man ét element i listen af gangen. 
Man sammenligner så dette element med det efterfølgende element, således, at den bytter plads med det næste element hvis dette er af en lavere værdi. 
Hvis elementet bytter plads med det næste element fortsætter man med at sammenligne med det efterfølgende element, hvis ikke sammenligner man det næste element med det der kommer efter det. 
Når så hele listen er gennemgået gør man det igen, men denne gang sammenligner man ikke med det sidste element i listen, da dette allerede vil være det største element. 
Dette gentager man n-1 gange, da man i det sidste skridt får de 2 første elementer på plads. 
Med Pseudokode kan denne algoritme skrives sådan:

\begin{algorithm}
\caption{Bubblesort}
\label{bubblesort}
\textbf{procedure} $bubblesort(a_1, a_2, ..., a_n   : $ reelle tal og hvor $n \geq 2$ \\
\textbf{for} $i:=1$ \textbf{til} $n-1$ \\
$\-$ $\-$ $\-$ $\-$ $\-$ $\-$
\textbf{for} $j:=1$ \textbf{til} $n-1$ \\
$\-$ $\-$ $\-$ $\-$ $\-$ $\-$
$\-$ $\-$ $\-$ $\-$ $\-$ $\-$
\textbf{hvis} $a_j>a_{j+1}$ \textbf{så} byt $a_j$ og $a_{j+1}$ \\
$\lbrace a_1, a_2, ..., a_n $ står nu i voksende rækkefølge $\rbrace $
\end{algorithm}