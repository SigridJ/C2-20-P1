\chapter{Kompleksitet}
Dette afsnit er skrevet med udgangspunkt i … , medmindre andet er angivet. \\ \\
For at man kan tale om, at en algoritme har løst et problem tilfredsstillende, er der to faktorer, der skal være opfyldt; den skal altid producere den rigtige løsning, og den skal være effektiv. 
Effektiviteten af en algoritme kan måles i den tid, det tager en computer at løse et problem ud fra algoritmen givet en bestemt mængde input, kaldet tidskompleksitet, samt den mængde hukommelse, algoritmen kræver på computeren, kaldet pladskompleksitet. 
Begge forhold bør tages stilling til, når algoritmer implementeres.
Det er vigtigt at vide, hvor lang tid det vil tage algoritmen at løse et problem, men også sørge for, at den krævede mængde hukommelse er tilgængelig på computeren, for at algoritmen kan implementeres. 
\\ En algoritmes tidskompleksitet er udtrykt ved antallet af operationer algoritmen udfører ved en bestemt mængde input, og ikke den faktiske computertid, da forskellige computer kræver forskellige tidsrum til at udføre operationerne i algoritmen. 
Antal operationer, algoritmen anvender, kan estimeres som en funktion af $n$, hvor $n$ er størrelsen på inputtet. 
For at vurdere, hvorvidt algoritmer er effektive til at løse problemer i takt med inputtet stiger, kan store-$O$ notationen benyttes.

\section{Store-$O$ notation}
Funktioners vækst kan beskrives ved store-$O$ notation. \\
\begin{defn}
	Lad $f$ og $g$ være funktioner fra mængden af heltal $\mathbb{Z}$ eller mængden af reelle tal $\mathbb{R}$ over i $\mathbb{R}$. Det hedder sig at $f(x)$ er $O(g(x))$ hvis der findes konstander $C$ og $k$ kaldet vidner, således 
\begin{align*}
\mid f(x) \mid \leq C \mid g(x) \mid
\end{align*}
så længe $x>k$.
\end{defn}

\begin{exmp}
Vis at $x^4+9x^3+4x+7$ er $O(x^4)$. \\
For at vise ovenstående skal der findes konstanterne $C$ og $k$, som opfylder $\mid f(x) \mid \leq C \mid g(x) \mid$ når $x>k$. $f(x)$ kan estimeres når $x>1$, hvoraf det følger $9x^3\leq 9x^4$ og $4x\leq 4x^4$ og $7\leq 7x^4$. Det følger derfor at 
\begin{align*}
x^4+9x^3+4x+7 &\leq x^4+9x^4+4x^4+7x^4 \\
&\leq 21x^4
\end{align*}
I eksemplet er vidnerne $C$ og $k$ bestemt til $C=21$ og $k=1$.
\end{exmp}

