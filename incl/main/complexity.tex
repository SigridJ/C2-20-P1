\chapter{Kompleksitet}
\section{Funktioners vækst}
Dette kapitel er skrevet med udgangspunkt i \citep{dmat}, medmindre andet er angivet. 

For at det kan konkluderes, at en algoritme har løst et problem tilfredsstillende, er der to faktorer, der skal være opfyldt. Den skal altid returnere den rigtige løsning, og den skal være effektiv. 
Effektiviteten af en algortime kan måles på flere måder.
Tidskompleksiteten af en algoritme betegner, hvor lang tid en computer tager om at køre algoritmen.
Pladskompleksiteten betegner på samme måde den mænge hukommelse, algoritmen bruger.
Både tidskompleksiteten og pladskompleksiteten er en måde at beskrive det antal recurser en algoritme bruger, hvilket også blot kaldes algoritmens kompleksitet.
Der bør tages stilling til begge forhold, når algoritmer implementeres.

Tiden den samme algoritme bruger på én computer er ikke nødvendigvis den samme på en andencomputer, da tiden også afhænger af en given computers computerkræft.
Derfor er en algoritmes tidskompleksitet udtrykt ved antallet af operationer der bliver udført, i stedet for den faktiske computertid.
Antal operationer, algoritmen anvender, kan estimeres som en funktion af størrelsen på inputtet. 
Til at vurdere, hvorvidt algoritmer er effektive til at løse problemer i takt med at inputtet stiger, kan store-$O$ notationen benyttes.

\subsection{Store-$O$ notation}
Funktioners vækst kan beskrives ved store-$O$ notation.
Ved brug af store-$O$ notation kan væksten af $f(x)$ sammenlignes med væksten af en funktion $g(x)$, for at undersøge om $f(x)$ vokser langsommere end $g(x)$ for $x$-værdier af en hvis størrelse.

\begin{defn}\label{eq_o}
	Lad $f$ og $g$ være funktioner fra $\Z \to \R$ eller $\R \to \R$.
	Det siges, at $f(x)$ er $O(g(x))$, hvis der findes konstanter $C$ og $k$, kaldet vidner, således 
	\begin{align*}
		|f(x)| \leq C |g(x)|
	\end{align*}
	så længe $x>k$.
\end{defn}

For at vise, at $f(x)$ er $O(g(x))$, er det tilstrækkeligt at finde ét par vidner $C$ og $k$, så ulighederne i Definition \eqref{eq_o} er opfyldt. Så længe der er ét par vidner, findes der uendeligt mange vidner. Er $C$ og $k$ vidner, så er $C'$ og $k'$ også vidner, hvis $C'>C$ og $k'>k$, da det så må gælde at $|f(x)| \leq C |g(x)| \leq C' |g(x)| $, så længe $x>k'>k$.

\begin{exmp}\label{exmp_bigO}
	Funktionen $f(x)=x^4+9x^3+4x+7$ er $O(x^4)$.
	Det kan vises ved at finde et par vidner $C$ og $k$, som opfylder $|f(x)| \leq C |g(x)|$ når $x>k$.
	Funktionen $f(x)$ kan estimeres når $x>1$, hvoraf det følger at $9x^3\leq 9x^4$ og $4x\leq 4x^4$ samt $7\leq 7x^4$.
	Det følger derfor, at 
	\begin{align*}
		f(x) &\leq x^4+9x^3+4x+7 \\
		&\leq x^4+9x^4+4x^4+7x^4 \\
		&\leq 21x^4.
	\end{align*}
	I eksemplet er vidnerne $C$ og $k$ fundet som $C=21$ og $k=1$, og $f(x)$ er derfor $O(x^4)$. 
\end{exmp}

Ofte kan potensfunktioner bruges til at estimere funktioners vækst, som set i Eksempel \ref{exmp_bigO}. 
I det følgende vil det blive vist, at polynomier af grad $n$ er $O(x^n)$.

\begin{thm}
Lad $f(x)=a_nx^n+a_{n-1}x^{n-1}+\cdots +a_1x+a_0$, hvor $a_0, a_1, \dotsc, a_{n-1}, a_n$ er reelle tal. 
Da er $f(x)$ $O(x^n)$.
\label{thm_poly}
\end{thm}

\begin{proof}
Antag, at $x>1$ og ved brug af trekantsuligheden, $|x| + |y| \geq |x + y|$, må det gælde at
	\begin{align*}
		|f(x)| &= |a_nx^n+a_{n-1}x^{n-1}+ \cdots +a_1x+a_0| \\
		&\leq |a_n|x^n + |a_{n-1}| x^{n-1}+ \cdots + |a_1| x +|a_0| \\
		&= |a_n| x^n + |a_{n-1}| x^nx^{-1}+ \cdots + |a_1| x^n x^{-n+1}+|a_0| x^nx^{-n} \\
		&= x^n \left(|a_n| + \frac{|a_{n-1}|}{x}+ \cdots +\frac{|a_1|}{x^{n-1}}+\frac{|a_0|}{x^n} \right) \\
		&\leq x^n(|a_n| + |a_{n-1}| + \cdots + |a_1| + |a_0| ).
	\end{align*}
Det er blevet bevist, at med vidnerne $C= |a_n| + |a_{n-1}| + \cdots + |a_1| + |a_0|$ og $k=1$, så er $f(x)=a_nx^n+a_{n-1}x^{n-1}+\cdots +a_1x+a_0$ $O(x^n)$.
\end{proof}

\subsection{Store-$\Omega$ notation}
Store-$O$ notation bruges i stort omfang til at beskrive væksten af funktioner og giver en øvre grænse af denne. 
For at beskrive en nedre grænse bruges store-$\Omega$ notation.
Ved brug af store-$\Omega$ notation kan væksten af $f(x)$ sammenlignes med væksten af en funktion $g(x)$, hvor der undersøges om $f(x)$ vokser hurtigere end $g(x)$ for $x$-værdier af en hvis størrelse.

\begin{defn}
	Lad $f$ og $g$ være funktioner fra $\Z \to \R$ eller $\R \to \R$. 
	Det siges, at $f(x)$ er $\Omega(g(x))$, hvis der findes konstanter $C$ og $k$, kaldet vidner, således 
	\begin{align*}
		|f(x)| \geq C |g(x)|
	\end{align*}
så længe $x>k$.
\end{defn}

\begin{exmp}\label{exmp_theta}
Funktionen $f(x)=x^4+9x^3+4x+7$ fra Eksempel \ref{exmp_bigO} er også $\Omega(x^4)$. 
Det ses nemlig, at $f(x)=x^4+9x^3+4x+7 \geq x^4 $ når $x>1$, og derfor er $f(x)$ $\Omega(x^4)$ med vidnerne $C=1$ og $k=1$.
\end{exmp}

\subsection{Store-$\Theta$ notation}
Store-$\Theta$ notation bruges, når der ønskes både en øvre og en nedre grænse for væksten af en funktion $f(x)$. \\

\begin{defn}\label{eq_theta}
	Lad $f$ og $g$ være funktioner fra $\Z \to \R$ eller $\R \to \R$. 
Funktionen $f(x)$ siges at være $\Theta (g(x))$, hvis $f(x)$ både er $O(g(x))$ og $\Omega (g(x))$. 
	Er $f(x)$ $\Theta (g(x))$ siges $f(x)$ at have samme orden som $g(x)$. 
\end{defn}

Ud fra Definition \ref{eq_theta} følger det, at $f(x)$ er $\Theta (g(x))$, hvis og kun hvis der findes konstanterne $C_1$, $C_2$ og $k$, således

\begin{align*}
	C_1 |g(x)| \leq |f(x)| \leq C_2 |g(x)|
\end{align*}

når $x>k$. \\
\begin{exmp}
Betragt igen funktionen $f(x)=x^4+9x^3+4x+7$ fra Eksempel \ref{exmp_bigO} og \ref{exmp_theta}. Denne er også  $\Theta(x^4)$.
Da $f(x)$ både er $O(x^4)$ og $\Omega (x^4)$, så er $f(x)$ også $\Theta (x^4)$.
\end{exmp}

\section{Kompleksitet af algoritmer}
Funktioners vækst kan beskrives ved store-$\Theta$ notation, hvor væksten af $f(n)$ estimeres af en funktion $g(n)$ med samme orden som $f(n)$. 
Hvis funktionen $f(n)$ beskriver antal operationer, en algoritme benytter for at udregne et resultat ved $n$ input, så angiver store-$\Theta$ notation tidskompleksiteten af algoritmen, hvor $\Theta(g(n))$ er ordenen. 
Der er forskellige typer kompleksitet, hvor den værst mulige kompleksiteten er algoritmens orden af det maksimale antal operationer en algoritme skal bruge for at udføre en opgave. 
Der findes også gennemsnitlig kompleksitet, der anvender det gennemsnitlige antal operationer for at løse et problem for alle mulige inputs.
Gennemsnitligt kompleksitet er mere kompliceret end den værst mulige kompleksitet, og derfor omtales kun den værst mulige kompleksitet i projektet.

I Tabel \ref{tab_complex} er algortimers typiske ordener og deres tilhørende kompleksitet angivet. 

\begin{table}[h]
 \begin{center}
  \begin{tabular}{|c|c|c|}
   \hline
   Kompleksitet & Terminologi\\
   \hline
   $\Theta(1)$ & Konstant kompleksitet\\
   $\Theta(\log n)$ & Logaritmisk kompleksitet\\
   $\Theta(n)$ & Lineær kompleksitet\\
   $\Theta(n \log n)$ & Linearitmisk kompleksitet\\
   $\Theta(n^b)$ & Polynomiel kompleksitet\\
   $\Theta(b^n)$, hvor $b>1$ & Eksponentiel kompleksitet\\
   $\Theta(n!)$ & Faktoriel kompleksitet\\
   \hline
  \end{tabular}
 \end{center}
 \caption{Terminologi for kompleksiteten af algoritmer.}
 \label{tab_complex}
\end{table}

I forbindelse med kompleksitet af algoritmer er funktionerne, der beskriver antal operationer, typisk af ordenen som de simple funktioner der ses i Tabel \ref{tab_complex}. 
En algoritme siges at have lineær værst mulig kompleksitet, hvis $f(n)$ beskriver det maksimale antal operationer,  algoritmen kan bruge, og $f(n)$ er $\Theta(n)$. 
En algoritme siges at have polynomiel værst mulig kompleksitet, hvis $f(n)$ er $\Theta(n^b)$, hvor at $f(n)$ er det maksimale antal operationer i algoritmen.

\begin{exmp}
Tidskompleksiteten af Algoritme \ref{find_maks} fra forrige kapitel, der finder det maksimale element i en begrænset mængde heltal, kan beskrives med store $\Theta$-notation. 
Antallet af sammenligninger bruges som måleenhed for tidskompleksiteten af denne algoritme, i det sammenligninger er den eneste basale operation, der bruges.

Det første element i listen sættes midlertidigt til $maks$, så $maks=a_1$. 
Ved en sammenligning $i \leq n$ fastslås det, hvorvidt listen fortsætter endnu. Hvis listen fortsætter, sammenlignes det midlertidige maksimum med det næste udtryk i listen. 
Hvis $maks<a_i$ opdateres det midlertidige maksimum, således $maks=a_2$.
Denne procedure bidrager med to sammenligninger for hvert udtryk i listen og endnu en for at gå ud af løkken, når $i=n+1$. 
I alt laves der $2(n-1)+1=2n-1$ sammenligninger, når denne algoritme kører. 
Derfor vil algoritmen til at finde det maksimale element i en mængde af $n$ elementer have lineær tidskompleksitet set i forhold til antallet af sammenligninger, der laves, fordi $f(n)=2n-1$ er $\Theta (n)$ jf. Sætning \ref{thm_poly}. 
\label{eks_lin_soeg} 
\end{exmp}

\section{Kompleksitetsklasser}
Problemer, der kan løses af algoritmer med polynomiel værst mulig tidskompleksitet, siges at tilhøre kompleksitetsklassen $P$. 
Algoritmer med polynomiel værst mulig tidskompleksitet kan forventes at finde en løsning til et problem inden for relativt kort tid. 
Polynomiets koefficienter og grad spiller dog en rolle, og derfor er polynomiel værst mulig tidskompleksitet ikke en garanti for, at en problem kan løses inden for en realistisk tidsramme.

Problemer, som ikke kan løses af algoritmer med  polynomiel værst mulig tidskompleksitet, siges at tilhøre klassen $NP$, hvis problemets løsning kan verificeres i polynomisk tid. Forkortelsen $NP$ står for “nondeterministisk polynonomiel” tid. 

En anden kompleksitetsklasse er $NP-fuldstændig$.
Det gælder for $NP-fuldstændige$ problemer, at alle $NP-fuldstændige$ problemer kan løses i polynomiel tid, hvis ét af de $NP-fuldstændige$ kan.
I $NP-fuldstædning$ klassen ligger kendte problemer som Boolean Satisfiabily Problem, Knapsack problem og at løse en sudoku eller et nonogram.
Der findes et kendt problem kaldet $P=NP$, som handler om at bevise (eller modbevise) at de to klasser er lig hinanden.
Problemet opstår fordi der tidligere har været problemer i $NP$ der pludselig er blevet fundet en løsning til, som placerer dem i $P$.
Det store spørgsmål er da, om dette altid er gældende.
