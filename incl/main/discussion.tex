\chapter{Diskussion}

Tidskompleksiteten, angivet ved Store-$\Theta$, estimerer tiden, det tager en algoritme at løse et problem, i takt med at inputtet stiger. 
Den faktiske computertid varierer, og nutidens hurtigste computere kan udføre en operation i en algoritme på $10^{-11}$ sekunder \citep{dmat}.
En brute-force algoritme til at løse TSP har faktoriel tidskompleksitet, $\Theta((n-1)!)$, hvor der estimeres en anvendelse af $(n-1)!$ operationer.
Med nutidens hurtigste computere, vil det tage algoritmen urimelig lang tid løse problemet, hvis antallet af knuder eksempelvis overstiger $n=20$, hvilket ses i Tabel \ref{tab_algtsp}. 
Det ses samtidigt, at Dobbelttræ-algoritmen har en væsentlig lavere gennemløbstid, og selv i større grafer vil algoritmen køres på under et millisekund. 

\begin{table}[h]
 \centering
  \begin{tabular}{|c|c|c|c|c|}
   \hline
   Algoritme & Kompleksitet & $n=10$ & $n=20$ & $n=30$\\
   \hline
		Brute Force & $\Theta((n-1)!)$ & $3,6 \cdot 10^{-6}$ sek & $14,1$ dage & $2,8 \cdot 10^{12}$ år \\
   \hline
   Dobbelttræ & $O(n^3)$ & $1,0 \cdot 10^{-9}$ sek & $8,0 \cdot 10^{-8}$ sek & $2,7 \cdot 10^{-7}$ sek \\
   \hline
  \end{tabular}
 \caption{Gennemløbstiden afhængig af antallet af knuder for løsningsalgoritmerne, hvor det er antaget, at der udføres én operation per $10^{-11}$ sekunder.} \label{tab_algtsp}
\end{table}

Grundet den lange gennemløbstid vil det være urentabelt at bruge ressourcer på at implementere en brute-force algoritme for at løse TSP i en større graf, selvom brute-force algoritmen med sikkerhed kan finde en Hamiltonkreds med mindst mulig vægt. 
I stedet kan der indgås et kompromis med algoritmens præcision. 

Ved at anvende Dobbelttræ-algoritmen findes der en approksimation på løsningen i stedet for den optimale løsning, men til gengæld forbedres gennemløbstiden markant, som det ses i Tabel \ref{tab_algtsp}. 
Den forbedrede hastighed for at køre algoritmen gør det muligt at approksimere løsninger af TSP, selv når mange byer skal besøges. 
Hvis approksimationen ikke medfører en uacceptabel stor fejl, kan det være en  fordel at løse problemet ved at gå på kompromis med algoritmens præcision.

Udover at være hurtig og præcis, set i forhold til at finde den optimale løsning, kan det overvejes, hvorvidt en algoritme skal kunne løse et givent problem i alle tilfælde. 
Der laves et kompromis, da TSP afgrænses til det metriske tilfælde, for at der eksisterer en approksimationsalgoritme.