\chapter{Diskussion}

Tidskompleksiteten angivet ved Store-$\Theta$ notation estimerer tiden, det tager en algoritme at løse et problem, i takt med at inputtet stiger. 
Den faktiske computertid varierer, og nutidens hurtigste computere kan udføre en operation i en algoritme på $10^{-11}$ sekunder.
En Brute Force algoritme til at løse Travelling Salesperson Problem har faktoriel tidskompleksitet, $\Theta(n!)$, hvor der estimeres en anvendelse af maksimalt $n!$ operationer.
Med nutidens hurtigste computere, vil det tage algortime urealistisk lang tid løse problemet, hvis inputtet når over eksempelvis $n=20$, hvilket ses i Tabel \ref{tab_algtsp}.

\begin{table}[h]
 \centering
  \begin{tabular}{|c|c|c|c|c|}
   \hline
   Algoritme & Kompleksitet & $n=10$ & $n=20$ & $n=30$\\
   \hline
   Brute Force & $\Theta(n!)$ & $3,6 \cdot 10^{-11}$ sek & $281,6$ dage & $8,4 \cdot 10^{13}$ år \\
   \hline
   Dobbelttræ & x & x & x & x \\
   \hline
  \end{tabular}
 \caption{Computertiden for at udføre løsningsalgortimerne til Travelling Salesperson Problem med $10^{-11}$ sekunder per operation.} \label{tab_algtsp}
\end{table}

Grundet den urealistisk lange computertid vil det ikke være rentabelt at bruge ressourcer på at implementere en Brute Force algortime for at løse Travelling Salesperson Problem i en større graf. 
Istedet kan der indgås et kompromis med algoritmens præcision. Ved at anvende Dobbelttræ algoritmen opnås der en approksimation på løsningen, men computertiden forbedres markant, som det ses Tabel \ref{tab_algtsp}.  
Udover egenskaberne at være hurtig at udføre og præcis set i forhold til at finde den optimale løsning kan det overvejes, hvorvidt en algoritme skal være kunne løse problemet for alle typer input.