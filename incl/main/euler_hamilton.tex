\section{Eulerkredse}

\begin{defn}\label{euler_def}
En Eulerkreds er en simpel kreds i grafen G som indeholder hver kant i G.
En Eulervej er en simpel vej i grafen G, som indeholder hver kant i G.  
\end{defn}

\noindent Det er ikke muligt at finde en Eulerkreds i alle grafer. 
Der findes en simpel måde at bestemme om der findes en Eulerkreds i en multigraf. 

\begin{thm}\label{Euler_multigraf}
En sammenhængende multigraf med mindst 2 knuder, har en Eulerkreds hvis og kun hvis, hver knude har en lige grad.
\end{thm}

\begin{proof} 
En kreds begynder en en knude $a$ og fortsætter langs en kant som en incident med $a$ til en nu knude $b$. 
Denne kant kaldes ${a,b}$, kanten bidrager med 1 til $deg(a)$. 
Hver gang kredsen passere gennem en knude, tilføjes 2 til graden af denne knude. 
Til sidst ender kredes tilbage i $a$, og bidrager igen med 1 til $deg(a)$. 
Derfor må $deg(a)$ være lige og graden af hver knude må også være lige.  
\end{proof} 

\noindent Det er muligt at lave en algoritme som kan finde en Eulerkreds i en multigraf, hvor hver knude er af lige grad.

En sådanne algoritme vil første tage udgangspunkt i en tilfældig delkreds i graf G. 
Hvorefter der indeførers en variabel H, som som en lig med G, foruden kanterne i den tilfældige kreds dannet i G. 
Så længe der stadig er kanter i H, til en løkke køre. 
Under denne løkke dannes en ny delkreds i H, en af knuderne skal også være endepunkt for en kant i en tidligere delkreds.
Dernæst fjerens kanterne i delkredsen fra H, sammen med enventuelt isolerede knuder. 
Til sidst sammensættes delkredsene til en kreds og denne kreds retuneres.
Dette kan ses i Algoritme \ref{algoritme_euler}.  

\begin{algorithm}
\caption{Eulerkredse}
\label{algoritme_euler}
\textbf{procedure} Euler(G: sammenhængende mulitgraf med knuder af lige grad)\\
$kreds:=$ en kreds i G begynder i en vilkårlig knude, med kanter der danner en kreds, som retunerer til startknuden.\\
$H:= G$ med kanterne fra $kreds$ fjernet\\
\textbf{når} $H$ har kanter\\
$\-$ $\-$ $\-$ $\-$ $\-$ $\-$
$delkreds:=$ en kreds i $H$ der begynder i en knude i $H$ som også er et endepunkt af en kant i $kreds$ \\ 
$\-$ $\-$ $\-$ $\-$ $\-$ $\-$
$H:=$ $H$ uden kanterne af delkredsen samt alle isolerede knuder fjernet \\
$\-$ $\-$ $\-$ $\-$ $\-$ $\-$
$kreds:=$ $kreds$ med $delkreds$ indsat ved den passende knude \\ 
\textbf{retuner} $kreds$ ($kreds$ er en Eulerkreds)
\end{algorithm}

Hvis der ikke findes en Eulerkreds i en graf, kan der godt eksistere en Eulervej. 

