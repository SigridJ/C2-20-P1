\section{The Halting Problem}
The Halting Problem går ud på, at man vil finde en procedure som kan bestemme, om en algoritme vil fortsætte uendeligt, eller om den stopper på et tidspunkt. 
Udfordringen består i, at denne løsning ikke findes, hvilket blev bevist af Alan Turring i 1936.
Grunden til at denne procedure ville være relevandt at finde er, at det er svært at vurdere, om en algoritme kører i uendelige løkker, eller om man blot skal vente lidt længere, før den er færdig.

\begin{proof}
Gå ud fra, at der er en løsning på Teh Halting Problem. Løsning kaldes $H(P, I)$. Proceduren tager to input; et program og et input til programmet. Programmet kan returnere to outputs, som enten kan være "ja", hvis programmet stopper, eller "nej", hvis programmet kører for evigt. 
Et program i sig selv kan også være et input til et program, og derfor kan input til H godt være P og P, så programmet P er input til sig selv.
H vil derfor bestemme, om P vil stoppe, når det får sig selv som input.

For at vise, at der ikke findes en procedure, som løser "The Halting Problem" introduceres et program. 
Programmet kaldes $K(P)$ og tager udskriften fra $H(P,P)$ som input.
Hvis udskrften fra $H(P,P)$ er 'ja' så vil $K(P)$ køre uendeligt, og det samme gælder omvendt, hvis 'nej' returneres vil K stoppe.  
Da et program kan tage sig selv som input, kan $K$ også have $K$ som input, hvilket kan sættes ind i $H$ og  medføre $H(K,K)$. 
$H(K, K)$ vil give to mulige outputs:
\begin{itemize}
	\item $K$ vil køre uendeligt, hvilket får $K(K)$ til at stoppe, og så vil $H(K,K)$ returnere "ja"
	\item  $K$ vil stoppe, hvilket får $K(K)$ til at køre uendeligt, og så vil $H(K,K)$ returnere "nej"
\end{itemize}
Begge disse muligheder modsiger sig selv, da K ikke både kan være endelig og uendelig. Derfor kan et program H ikke eksistere, da det ikke kan svare rigtigt på, om alle programmer kan køre uendeligt. 

\end{proof}
