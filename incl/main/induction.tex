\chapter{Induktionsbeviser}

Dette kapitel tager udgangspunkt i \citep{dmat}, medmindre andet er angivet.

Induktionsbeviser er en type af beviser, som typisk er nyttige til at vise diskrete sammenhænge.
Det kan eksempelvis være sætninger om naturlige tal, algoritmer eller grafer.

Logikken bag induktionsbeviset gør brug af en type funktion kaldet \textit{et åbent udsagn}.
Et udsagn kan navngives $P$, hvor $P(x)$ nu er det åbne udsagn som enten får værdien sand eller falsk når $x$ bliver tildelt en værdi.
Et åbent udsagn kan da være sandt for eksempelvis alle $x$ der tilhører et domæne $D$.

\begin{theorembox}{Princippet om matematisk induktion}
	De to følgende skridt skal gennemføres for at bevise, at $P(n)$ er sand for alle $n \in \N$, hvor $P(n)$ er et åbent udsagn.

	\textbf{Basisskridtet:} \quad 
	Vis at $P(1)$ er sand.
	
	\textbf{Induktionsskridtet:} \quad 
	Vis, at hvis $P(k)$ er sand, så følger det at $P(k + 1)$ er sand, for et vilkårligt $k \in \N$.
\end{theorembox}

I induktionsskridtet antages det, at $P(k)$ er sand for et vilkårligt $k$.
Denne antagelse kaldes induktionsantagelsen.
Det kan umiddelbart ligne et "cirkelbevis", fordi der antages, hvad det i sidste ende er målet at vise.
Der er dog forskel på at antage en sætning for at vise denne, og at antage en sætning for at vise, at en anden følger heraf.

En intuitiv måde at betragte induktionsteknikken er ved at tænkte på en række af dominobrikker.
Hvis den første brik væltes, og det er sikkert, at hvis en vilkårlig af brikkerne vælter, så vælter den næste også, så må det være sandt, at hele rækken vælter.
Induktionsskridtet kan altså betragtes som en dominoeffekt.
Desuden er det også nemt at forestille sig, at hvis rækken af dominoer var uendelig lang, så vil en vilkårlig brik i rækken blive væltet hvis den første væltes. 

Induktionsbeviser kan blandt andet bruges til at bevise sætninger om summer af de naturlige tal.
\begin{exmp}
	Det vil med induktion vises, at $1 + 3 + 5 + \dotsb + (2n-1) = n^2$ for alle $n \in \N$.

	Lad udsagnet $P$ være, at $1 + 3 + 5 + \dotsb + (2n-1) = n^2$.
	
	\textit{Basisskridt:} $P(1)$ er sand, både da $1^2 = 1$ og summen når $n = 1$ er lig $1$.

	\textit{Induktionsskridt:} Antag at $P(k)$ er sand, for et vilkårligt $k \in \N$.
	Det skal nu vises, at det følger, at $P(k + 1)$ da også må være sand. 
	Hvis
	\begin{align}
		1 + 3 + 5 + \dotsb + (2k-1) 
		= k^2, \nonumber
	\end{align}
	så gælder det, at
	\begin{align}
		1 + 3 + 5 + \dotsb + (2k-1) + (2k+1) 
		&= k^2 + (2k + 1) \nonumber \\
		&= k^2 + 2k + 1 \nonumber \\
		&= \left( k + 1 \right) ^2. \label{eq:eks1_indu}
	\end{align}
	Her ses det, at $P(k + 1)$ netop er Ligning \eqref{eq:eks1_indu}, hvilket betyder, at $P(k + 1)$ er sand, når $P(k)$ er sand, hvilket afslutter induktionsskridtet.

	Da både basisskridtet og induktionsskridtet er gennemført, må det betyde, at $1 + 3 + 5 + \dotsb + (2n-1) = n^2$ for alle $n \in \N$.
\end{exmp}

\begin{wrapfigure}{R}{4cm}
	\centering
	\includegraphics[scale=0.21]{fig/img/sum_of_n_first_odd_integers.png}
	\caption{En illustration over hvordan man kan arrangere de første $n$ ulige tal som et kvardrat.} \label{fig1_indu}
\end{wrapfigure}

Induktionsbeviser er meget stærke men adskiller sig fra andre bevistyper som direkte beviser, modstridsbeviser eller modeksempler ved, at selve beviset umiddelbart ikke skaber større forståelse for, \textit{hvorfor} sætningen er sand på samme måde, som de andre kan.

I Eksempel 4.1 blev det vist, at summen af de første $n$ ulige naturlige tal er lig $n^2$, men selve beviset bragte ikke større indsigt i, hvorfor denne sammenhæng er sand.
Et eksempel på en indsigt om sammenhængen er illustreret i  Figur \ref{fig1_indu}.
Figuren viser hvordan man altid kan arrangere summen af de første $n$ ulige tal i et kvardrat, og denne simple figur kan være del af et direkte bevis for sætningen.
Dette bevis vil meget bedre kunne illustrere sammenhængen end det tideligere induktionsbevis.

\section{Velordningsprincippet}
Induktionsbeviset bygger på en fundamental egenskab ved de naturlige tal kaldet velordningsprincippet.
Det er et aksiom for de naturlige tal og er beskrevet som følgende.
\begin{theorembox}{Velordningsprincippet}
	Enhver delmængde af $\N$, der ikke er tom, har et mindste element.
\end{theorembox}
\noindent Som resultat af velordningsprincippet følger det også, at alle delmængder af de naturlige tal har en rækkefølge fra det mindste element til det største, og at "det næste element i rækken" er veldefineret, da det netop er det element, der er det mindste i mængden fraregnet alle de foregående elementer.

Velordningsprincippet gør det muligt at bevise induktionsbeviset er gyldig.

\begin{thm}
	Induktionsprincippet følger af velordningsprincippet.
\end{thm}

\begin{proof}
	Antag at $P(1)$ er sand, og at $P(k) \Rightarrow P(k + 1)$ er sand for alle $k \in \N$.
	Det skal nu vises at $P(n)$ da også må være sand for alle $n \in \N$.

	Antag for modstrid at $P(n)$ er falsk for mindst ét $n$.
	Lad mængden $S$ være mængden af alle $n$ hvor $P(n)$ er falsk.
	Derfor må $S \subseteq \N$ ikke være tomt, og det følger af velordningsprincippet, at den har et mindste element $s$.
	Da $P(1)$ er sand må $s \neq 1$, og $s - 1$ tilhører da $\N$.
	Eftersom $s$ er det mindste element i $S$, må $P(s - 1)$ være sand, og da $P(k) \Rightarrow P(k+1)$ er sand, må $P(s)$ da også være sand.
	At $P(s)$ både er sand og falsk er en modstrid.

	Det må derfor gælde at $(P(1) \land \forall k ( P(k) \Rightarrow P(k + 1))) \Rightarrow \forall n P(n)$, hvilket betyder at induktionsbeviset er gyldigt.
\end{proof}

Udover at bevise sætninger om summer på naturlige tal, kan induktionsbeviser også bruges til at bevise sætninger om grafer.
Gennem projektet vil der blive brugt induktionsbeviser til netop dette.
