\chapter{Indledning}
Et kendt problem inden for diskret matematik er Travelling Salesperson Problem, der omhandler, hvordan der kan findes en kortest mulig vej mellem et antal byer. 
Det er tanken, at en handelsrejsende skal besøge hver by én gang og ende i samme by, og ruten skal optimeres, så distancen er mindst mulig. 
Et vejnetværk mellem byer kan repræsenteres ved hjælp af grafteori, og ved hjælp af algoritmer kan en kortest mulig kreds i en graf bestemmes. 
Algoritmer er ikke alle lige effektive til at løse problemer, og deres effektivitet kan beskrives ved af tidskompleksiteten. 
Tidskompleksiteten er et udtryk for, hvor hurtigt en algoritmes gennemløbstid vokser, i takt med inputtet stiger. 

Mange virkelige problemstillinger kan relateres til Travelling Salesperson Problem eksempelvis i forbindelse med firmaers ønske om at optimere produktioner eller transportruter. 
Der findes ikke en effektiv algoritme, der kan løse Travelling Salesperson Problem inden for en rimelig tidsramme, og det vil derfor til sidst i projektet blive diskuteret, hvorvidt der kan gås på kompromis med præcisionen og i stedet approksimere en løsning til problemet ved hjælp approksimationsalgoritmer.
\\
Ovenstående analyse leder frem til følgende problemformulering:\\

\textit{Hvad er algoritmer, kompleksitet, induktionsbeviser og grafer? Hvordan kan Travelling Salesperson Problem repræsenteres ved hjælp af grafteori, og hvorledes kan et løsningsforslag til Travelling Salesperson Problem approksimeres?}
\\\\

