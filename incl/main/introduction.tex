\chapter{Indledning}
Et stort problem indenfor matematikken er Travelling Salesperson Problem. 
Dette er et problem, der handler om at finde den kortest mulige kreds i en graf. 
For at undersøge dette problem, skal der bruges toeri om blandt andet algoritmer, kompleksitet og grafer.
Algoritmer bruges til at løse matematiske problemer via en række af skridt der skal genneføres. 
Algoritmer skal bruges for at komme med forslag til en løsningsalgoritme til Travelling Salesperson Problem. 
For at finde ud af, hvor kompliceret en algoritme er, udregner man algoritmens kompleksitet. 
Dette bruges, til at undersøge hvor effektiv en løsningsalgoritme til Travelling Salesperson Problem er. 
Grafer bruges til at vise sammenhænge. 
I tilfældes med Travelling Salesperson Problem, bruges grafer til at vise, hvordan byer og steder ligger i forhold til hinanden og hvordan de er forbundet. 
I dette projekt vil der blive fokuseret meget på teorien der ligger bag Travellign Salesperson Problem.\\

\noindent \textbf{Problemformulering} \\
Hvordan kan teori om algoritmer, kompleksitet og grafer bruges til at løse Travelling Salesperson Problem?
\\\\
Hvad er algoritmer, kompleksitet, induktionsbeviser og grafer? Hvordan kan Travelling Salesperson Problem repræsenteres ved hjælp af grafteori, og hvorledes kan et løsningsforslag til Travelling Salesperson Problem approksimeres?
\\\\

