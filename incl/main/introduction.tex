\chapter{Indledning}
Et kendt problem inden for diskret matematik er Travelling Salesperson Problem, også kendt som TSP, der omhandler, hvordan der kan findes en kortest mulig vej mellem et antal byer. 
Det er tanken, at en handelsrejsende skal besøge hver by én gang og ende i samme by, og ruten skal optimeres, så distancen er mindst mulig. 
Et vejnetværk mellem byer kan repræsenteres ved hjælp af grafteori, og ved hjælp af algoritmer kan en kortest mulig kreds i en graf bestemmes. 
Algoritmer er ikke alle lige effektive til at løse problemer, og deres effektivitet kan beskrives ved af tidskompleksiteten. 
Tidskompleksiteten er et udtryk for, hvor hurtigt en algoritmes gennemløbstid vokser, i takt med inputtet stiger. 

Mange virkelige problemstillinger kan relateres til TSP eksempelvis i forbindelse med firmaers ønske om at optimere produktioner eller transportruter. 
Gennem projeket vil der blive undersøgt hvorvidt der findes en effektiv algoritme der kan løse TSP inden for en rimelig tidsramme.
Ovenstående analyse leder frem til følgende problemformulering:\\

\textit{Hvad er algoritmer, kompleksitet, induktionsbeviser og grafer? Hvordan kan Travelling Salesperson Problem repræsenteres ved hjælp af grafteori, og hvorledes kan en løsning til Travelling Salesperson Problem approksimeres ved hjælp af algoritmer? Diskutér løsningsalgoritmernes kompleksitet.}
