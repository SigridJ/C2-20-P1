\section{Introduktion til TSP}
Grafer kan bruges til at vise mange forskellige ting. 
Det kan eksempelvis være netværk, sociale relationer eller vejnet. 
I en graf som viser et vejnet, vil steder blive vist med knuder og vejene imellem disse bliver vist med kanter.
Somme tider ønskes det, at finde en vej eller kreds i en graf, det kan være i forbindelse med ruteplanlægning, overførsel af data i et netværk eller lignende. 
I grafer over eksempelvis vejnet er det meget relevant at bruge vægtede grafer, da det er nyttigt at kunne se hvor lang en strækning mellem 2 steder er.
Dette kan så bruges til at finde den korteste rute eller bare undersøge længden af en bestemt rute. 
Der er mange problemer der omhandler veje og kredse i vægtede grafer, et af disse er Travelling Salesperson Problem.
Travelling Salesperson Problem, også kaldet som TSP, går ud på, at en handelsrejsende vil finde den korteste rute gennem alle steder han vil besøge, hvor han til sidst ender tilbage hvor han startede.
Det vil altså sige, at det ønskes at finde en simpel kreds i en vægtet graf, der går gennem alle punkter en gang. 
Den kreds der skal findes kan beskrives som en hamiltonkreds, fordi den skal gå gennem hver knude netop én gang. 