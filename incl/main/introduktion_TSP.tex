\section{Introduktion til Travelling Salesperson Problem}
Grafer kan bruges til at vise mange forskellige ting som eksempelvis netværk, sociale relationer eller vejnet. 
I en graf som viser et vejnet, vil steder blive vist med knuder og vejene imellem vises med kanter.
Sommetider ønskes det at finde en vej eller kreds i en graf.
Det kan være i forbindelse med ruteplanlægning, overførsel af data i et netværk eller lignende. 
I grafer over eksempelvis vejnet er det meget relevant at bruge vægtede grafer, da det er nyttigt at kunne se, hvor lang en strækning mellem to steder er.
Det kan bruges til at finde den korteste rute eller bare undersøge længden af en bestemt rute. 

Der er mange problemer, der omhandler veje og kredse i vægtede grafer, hvor et eksempel er TSP.
Problemet går ud på, at en handelsrejsende vil finde den korteste rute gennem alle steder, vedkommende vil besøge, og til sidst ende tilbage hvor rejsen startede.
Det vil specielt sige, at det ønskes at finde en simpel kreds i en vægtet graf, som er en Hamiltonkreds, fordi den skal gå gennem hver knude netop én gang. Kapitlet tager udgangspunkt i \citep{metrictsp} medmindre andet er angivet.

TSP er et populært, svært problem og er kendt for, at der ikke findes nogen algoritmer til at løse optimeringsproblemet i polynomisk tid. 

\begin{tcolorbox}
	\textbf{Travelling Salesperson Problem} \quad Lad grafen $G=(V,E)$ være en simpel komplet graf med vægtede kanter.

	Find en Hamiltonkreds i $G$ med den mindste samlede vægt.
\end{tcolorbox}