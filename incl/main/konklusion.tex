\chapter{Konklusion}


I løbet af dette projekt er det blevet undersøgt, hvad algoritmer, kompleksitet, induktionsbeviser og grafer er.
 
Algoritmer er en nødvendig sekvens af skridt, som kræves for at løse et matematisk problem. 
Der findes mange forskellige former for algoritmer, blandt andet brute-force algoritmer og grådige algoritmer. 
For algoritmer er det relevant at se på, hvor tids- og ressourcekrævende de er, hvilket beskrives ved kompleksitet. 
I projektet er Store-$O$, Store-$\Omega$ og Store-$\Theta$ notation blevet undersøgt. 
Notationerne kan bruges til at undersøge en algoritmes tidskompleksitet.
 
I projektet er induktionsbeviser også blevet belyst, idet de er nyttige til at bevise sætninger indenfor diskret matematik, herunder grafteori. 
Induktionsbeviser indeholder to skridt: basisskridtet og induktionsskridtet.
Skridtene bruges til at undersøge, om et åbent udsagn gælder for alle værdier, der kan indsættes. 

Indenfor grafteori er det blevet beskrevet, hvad grafer er, samt hvilke anvendelser de har. 
I den sammenhæng er der blevet defineret flere forskellige grafer, herunder  simple grafer, sammenhængende grafer, multigrafer, vægtede grafer og træer. 
Grafer kan indeholde veje og kredse, som gennemløber kanter og knuder. 
De eksempler, der er i fokus, er Euler- og Hamiltonkredse, da de er relevante for TSP. 
Hamiltonkredse er specielt vigtige, da det er simple kredse, der besøger hver knude netop én gang, og da er den kortest mulige Hamiltonkreds løsningen til TSP. 

Det er blevet vist, at den generelle version af TSP ikke kan approksimeres i polynomiel tid. Derfor er det nødvendigt at restringere TSP til eksempelvis den metriske version.

I projektet er to løsningsalgoritmer til TSP blevet undersøgt. 
Den første algoritme der beskrives er brute-force algoritmen, der har en meget høj tidskompleksitet, men som altid finder den optimale løsning. 
Løsningen findes ved at sammenligne alle mulige Hamiltonkredse og dermed vælge den kortest mulige kreds.  
Den anden algoritme er Dobbelttræ-algoritmen, som er en approksimationsalgoritme til den metriske version af TSP. 
Den finder ikke nødvendigvis den optimale løsning til TSP, men en $2$-approksimation. 
Begge algoritmer har deres fordele og ulemper, og valget mellem dem går på, om gennemløbstid eller præcision vægtes højst. 