\chapter{Konklusion}


I løbet af dette projekt, er det blevet undersøgt hvad algoritmer, kompleksitet, induktionsbeviser og grafer er. 
Algoritmer er en nødvendig sekvens af skridt, som kræves for at løse et matematisk problem. 
Der findes mange forskellige former for algoritmer, blandt andet brute-force algortimer, der er en type af algoritmer der slavisk går gennem en masse skridt. 
For algortimer er det relevandt at se på, hvor tids- og ressourcekrævende de er, til dette anvendes kompleksitet. 
I dette projekt er store-O, store-$\Omega$ og store-$\Theta$ notation blevet undersøgt. 
Disse notationer kan bruges til at undersøge en algoritmes tidskompleksitet. 
I projektet er induktionsbeviser også blevet undersøgt, fordi de er nyttige til at bevise sætninger indenfor diskret matematik, herunder grafteori. 
Om induktionsbeviser er det blevet fundet, at de indeholder to skridt, basisskridtet og induktionsskridtet.
Disse bruges til at undersøge, om en induktionsantagelse gælder for alle værdier der kan indsættes. 
Indenfor grafteori, er det blevet undersøgt, hvad en graf er, samt hvilke anvendelser der er deraf. 
I denne sammehæng er der blevet set på flere forskellige grafer og deres egenskaber, herunder  simple, sammenhængende grafer, multigrafer, vægtede grafer og træer. 
Grafer kan somme tider indeholde veje og kredse, som går gennem kanter og knuder. 
De eksempler der her er blevet undersøgt er euler- og hamiltonkredse, da disse ligger tæt op ad Travelling Salesperson Problem. 
Hamiltonkredse er specielt vigtige, da det er en simpel kreds, der besøger hver knude netop en gang og ikke bruger den samme kant flere gange.
På den måde kan den korteste hamiltonkreds siges at være løsningen til TSP. 
I projektet er løsningsalgoritmer til Travelling Salesperson Problem blevet undersøgt. 
Den første algoritme der undersøges er bruteforce algortime, der har en meget høj tidskompleksitet, hvis grafen har mange knuder. Denne type ender dog med at finder den optimale løsning i sidste ende. Dette gør den ved at sammenligne alle mulige løsninger og dermed vælger den kortest mulige vej.  
Den anden algortime er dobbelttræalgoritmen, som er en approksimationsalgoritme som ikke finder den optimale løsning til TSP, men altså en approksimation som nogle lunde er brugbar. 
Hver af disse algoritmer har deres fordele og ulemper, og valget mellem dem går på, om gennemløbstid eller præcision vægtes højest. 
