\section{Løsningsalgoritmer}

Der findes indtil flere løsningsalgoritmer til at løse TSP. Det viser sig dog imidlertid, at der ikke er nogen let måde, at løse problemet på. Som tidligere nævnt, er det nødvendigt at ty til approksimationsalgoritmer for at have en chance for at finde en nogenlunde brugbar løsning.

Den mest umiddelbart løsningsalgoritme til at finde en korteste Hamiltonkreds rundt i en graf, finder samtlige mulige Hamiltonkredse i grafen. Til hver af disse kredse beregnes den samlede vægt, og til sidst sammenlignes alle disse samlede vægte, for at finde den Hamiltonkreds, som har den mindste samlede vægt. Denne algoritme er i pseudokode beskrevet i Algoritme \ref{brute_force}.

\begin{algorithm}
\caption{Brute-force algoritmen}
\label{brute_force}
\textbf{procedure} Find korteste Hamiltonkreds i grafen $G$ med $\alpha$ Hamiltonkredse.

Bestem alle Hamiltonkredse, $H_k$, i $G$, og lad $H_{ki}$ betegne den $i$'te Hamiltonkreds. \\
\textbf{for} $i:=1$ til $\alpha$ \\
$\-$ $\-$ $\-$ $\-$ $\-$ $\-$
Bestem $d(H_{ki})$ \\
$min := H_{k1}$ \\
\textbf{for} $i:=2$ til $\alpha$ \\
$\-$ $\-$ $\-$ $\-$ $\-$ $\-$
\textbf{hvis} $min > H_{ki}$ \textbf{så} $min := H_{ki}$ \\
\textbf{returnér} $min$ $\lbrace$ $min$ er den korteste Hamiltonkreds i $G$ $\rbrace$
\end{algorithm}

Denne algoritme vil altid finde den korteste Hamiltonkreds i en komplet graf $G$, men kræver meget processorkraft at gennemføre, selv for grafer med relativt få knuder.

\begin{thm}
Den værst mulige tidskompleksitet af brute-force algoritmen er $O(n!)$.
\end{thm}

\begin{proof}
Lad $G$ være en komplet graf med $n$ knuder og $n \geq 3$. I denne graf er der så $(n-1)!$ forskellige Hamiltonkredse, idet de $n-1$ punkter kan arrangeres på $(n-1)!$ forskellige måder. Derfor har algoritmen allerede her en værst mulig tidskompleksitet på $O(n!)$ idet $(n-1)!$ er $O(n!)$. 

Herfter foretager algoritmen en lineær søgning, som jf. Eksempel \ref{eks_lin_soeg} har en værst mulig tidskompleksitet på $\Theta(n)$, og spiller derfor ingen rolle i algoritmens samlede kompleksitet.
\end{proof}

Brute-force algoritmen har således en kompleksitet, der er voldsom stor selv ved relativt få knuder i $G$.

\begin{exmp}
I den komplet graf, $G$, med $20$ knuder, har brute-force algoritmen en værst mulig tidskompleksitet på $$20! = 2432902008176640000,$$ hvorfor algoritmen tydeligvis allerede ved $20$ knuder er håbløs.
\end{exmp}

Derfor er approksimationsalgoritmer nødvendige, for at løse problemet. Som tidligere beskrevet, vil disse algoritmer \textit{ikke} finde den optimale løsning, men en løsning, som ligger indenfor en vis konstant grænse. I denne rapport vil Dobbelttræ-algoritmen anvendes til at løse TSP. 

\subsection{Dobbelttræ-algoritmen}
Dobbelttræ-algoritmen forudsætter, at grafen, som Hamiltonkredsen skal findes i, er komplet metrisk. For sådanne grafer kan der imidlertid foretages en \textit{genvej}, som defineres i Definition \ref{def_genvej}.

\begin{defn}
Lad $G$ være en komplet, metrisk graf med $n$ knuder og $n \geq 3$. I $G$ findes en vej, $P = v_0, v_1,...,v_{i-1}, v_i, v_{i+1},...,x_{n-1},x_n$ hvor $1 \leq i \leq n-1$. \\
I $P$ findes så to kanter $e_i = \lbrace x_{i-1}, x_i \rbrace$ og $e_{i+1} = \lbrace x_i, x_i+1 \rbrace$, så $e_i \cap e_{i+1} = x_i$.
Der findes så en trekant $\Delta_i = \lbrace e_i, e_i+1, e_s \rbrace$ i $G$, hvor $e_s = \lbrace x_{i-1}, x_{i+1} \rbrace$. \\
Da kan der dannes en genvej i $P$ via $e_s$ således en ny vej, $P'=v_0, v_1,...,x_{i-1},x_{i+1},...,x_{n+1},x_n$
\label{def_genvej}
\end{defn}




