\chapter{Perspektivering}
I projektet er det kun optimeringsprobelemet af TSP der er blevet undersøgt. Dog er der flere typer af algoritmiske problemer med udspring i TSP. 
Søges der efter flere korrekte svar, kaldes problemet et søgningsproblem. Søges en optimal løsning, kaldes problemet et optimeringsproblem. 
Optimeringsproblemer og søgeproblemer kaldes også evalueringsproblemer eftersom evalueringsproblemer altid har unikke løsninger.

En anden type algoritmiske problemer er beslutningsproblmer, hvor der blot søges ét af to svarmuligheder, ja eller nej. 
Denne version af TSP tager endnu et input, netop en samlet vægt, og der skal så tjekkes, om der eksisterer en Hamiltonkreds i grafen med mindre samlet vægt end denne.

Forskellige versioner af TSP kan have forskellige løsningsalgoritmer. 
Specielt er beslutningsversionen af TSP $NP$-fuldstændigt \citep{complex}.

TSP kan perspektiveres til Chinese Postman Problem, som blev fremsat i 1960'erne af den kinesiske matematiker Mei-Ko Kwan. 
Problemet går ud på, at en postbud skal levere post, og vedkommende ønsker at gå den korteste distance. 
Postbuddet skal gå langs alle veje i området, da der formenligt skal leveres post til alle husstande, og buddet starter og slutter ved posthuset. 
Modsat salgspersonen i TSP, skal postbuddet forbi alle veje i et vejnet for at kunne levere post, mens salgspersonen i TSP, kun skal finde den korteste vej fra by til by. 
Vejnettet, som postbuddet skal benytte sig af, kan omdannes til en vægtet graf, hvor vægtene kan repræsentere længden af hver vej. 
Det er nemt at udregne længden af postbuddets rute, hvis alle grafens knuder er af lige grad, eftersom der så eksisterer en Eulerkreds. 
Eksistensen af en Eulerkreds medfører, at hver kant kun skal passeres præcis én gang, hvorfor det er underordnet hvilken rute der vælges.   
Indholder grafen et lige antal knuder af ulige grad, betyder det at der ikke længere eksisterer en Eulerkreds, og buddet skal dermed gennem flere kanter mere end én gang. 
For at bestemme den kortest mulige rute, kan der gøres brug af Chinese Postman-algoritmen, som har polynomiel tidskompleksitet  \citep{dmat}.