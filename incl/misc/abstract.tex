% incl/misc/abstract.tex : projektets abstract
% ------------------------------------------------------------------------------
% Et abstract er et kort resume af rapporten, som vises på titelbladet

In discrete mathematics, the famous problem Travelling Salesperson Problem (TSP) states the task of optimizing a route between different cities in terms of travelled distance. 
When applying concepts from graph theory, solving TSP is equivalent to finding the shortest possible simple circuit through all vertices in a complete weighted graph, which is called a Hamilton circuit. 
For solving many mathematical problems, algorithms can be implemented. Algorithms are precisely defined steps that together make up the calculations for finding a solution. 
Not all algorithms are equally efficient. For making it possible to compare different algorithms, Big-O, Big-Omega and Big-Theta notations are introduced together with the term time complexity. 
Algorithms with polynomial time complexity can solve a problem relatively fast. 
However, it is not possible to solve TSP with a polynomial algorithm, which means finding an optimal solution can be an unreasonably slow task for a larger number of cities. 
Instead, a solution to TSP can be estimated by approximation algorithms.
Even though they lack precision, these algorithms can give a solution to TSP in polynomial time. 
